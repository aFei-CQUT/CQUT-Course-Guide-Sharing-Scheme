%% 任务书
\begin{taskbook}
	\taskinfo		% or use `\taskinfo*' for less lines.
	
	
	
	%%%  1.设计(论文)的主要任务及目标
	\taskitem
	
	按不同学号给定的任务参数如下表:
	
	\begin{table}[H]
		\centering
		\caption*{按不同学号给定的吸收填料塔参数表}
		\begin{tabular}{|c|c|c|c|c|}
			\hline
			\ce{SO2}吸收率 & \multicolumn{3}{c|}{混合气体处理量 (m³/h)} & 操作条件 \\
			\hline
			& 900 m³/h & 1000 m³/h & 1100 m³/h & \\
			\hline
			96 \% & 4-6 & 25-27 & 16-18 & 常压 \\
			\hline
			98 \% & 19-21 & 1-3 & 34-36 & 绝压120kPa \\
			\hline
			96 \% & 10-12 & 31-33 & 7-9 & 常压 \\
			\hline
			98 \% & 28-30 & 13-15 & 22-24 & 绝压120kPa \\
			\hline
			& 5.5\% & 6\% & 6.5\% & \\
			\hline
			& \multicolumn{3}{c|}{进塔气体中\ce{SO2}的摩尔分数} &  \\
			\hline
		\end{tabular}
	\end{table}
	
	依据学号查得个人任务如下:
	
	\textbf{设计一个\ce{SO2}吸收填料塔,在常温,绝压$\mathbf{120kPa}$操作条件下,处理气体流量为$\mathbf{1000m³/h}$,\ce{SO2}含量为$\mathbf{6.0\%}$的废气,要求\ce{SO2}吸收率大于等于$\mathbf{98\%}$,使用聚丙烯散装鲍尔环作为填料。}



	%%% 2.设计(论文)的基本要求和内容
	\taskitem
	1. 选择合适的填料规格;
	2. 计算所需填料塔的尺寸和高度;
	3. 确定吸收液(\ce{H2O})的循环量;
	4. 分析操作条件对吸收效率的影响;
	5. 提供详细的设计图纸和流程图;
	6. 编写设计说明书,包含计算过程和结果分析。
	\clearpage



	%%% 3.主要参考文献
	\taskitem
	\begin{bibenumerate}
		\item 刘海洋. \LaTeX\ 入门\cite{latexrumen} [M]. 北京 : 电子工业出版社, 2013.
		\item MITTELBACH F, GOOSSENS M, BRAAMS J, et al. The \LaTeX\ Companion[M]. 2nd ed. Reading, Massachusetts : Addison-Wesley, 2004.
	\end{bibenumerate}
	
	
	
	%%% 4.进度安排
	\taskitem
	\begin{table}[H]
		\centering
		\begin{tabularx}{.95\textwidth}{p{1.5em}|X|p{6em}}
			\hline
			& 设计(论文)各阶段名称 &	起止日期\\
			\hline
			1 & 文献调研与方案设计 &	2024-09-07 至 2024-09-08\\
			\hline
			2	& 填料塔尺寸计算 &	2024-09-09 至 2024-09-11\\
			\hline
			3 & 设计图纸绘制与说明书编写 & 2024-09-12 至 2024-09-20\\
			\hline
		\end{tabularx}
	\end{table}	
\end{taskbook}