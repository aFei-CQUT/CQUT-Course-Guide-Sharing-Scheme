\begin{foreword}
	
	二氧化硫(\ce{SO2})作为一种主要的大气污染物,不仅会导致酸雨的形成,还会对人体健康造成严重危害,诱发呼吸系统疾病、视觉系统疾病等。如今,随着工业化进程的不断加快,\ce{SO2}排放量逐年递增,控制和治理二氧化硫成为环境保护和可持续发展的重要课题。
	
	在化工领域中处理废气的方式多种多样,吸收是多种方式中行之有效的一种,其基本思想是利用混合气体各组分在液体溶剂中溶解度的不同来达到分离混合气体各组分的目的。如何高效地利用吸收操作去除工业废气中的\ce{SO2},是亟待解决的问题之一。
	
	一般来说,完整的吸收过程应包括吸收和解吸两部分。在化工生产过程中,原料气的净化、气体产品的精制、治理有害气体、保护环境等方面都要用到吸收将气体混合物中的各个组分加以分离,其目的是:①回收或捕获气体混合物中的有用物质,以制取有价值的产品;②除去混合气体中的有害成分,使气体净化,以便进一步加工处理;或除去工业放空尾气中的有害物质,以免污染大气。
	
	考虑到\ce{SO2}具有腐蚀性,若采用板式塔结构\ce{SO2}会对塔板造成严重的腐蚀影响吸收塔的正常工作,故此次设计应选择填料塔设计。此外,填料塔结构简单、操作弹性大、压降较低、能耗较小、传质效率高、分离效率高,更适合大规模工业应用,治理废气。
	
	二氧化硫吸收填料塔,以水为溶剂,经济合理,净化度高,污染小。由于水和二氧化硫反应生成硫酸,同时也具有相当的利用价值。
	
\end{foreword}