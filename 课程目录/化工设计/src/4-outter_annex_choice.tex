%%% 第四章
\chapter{吸收塔外部附件的选择}

%%% ===============================================
\section{吸收塔主要接管的尺寸计算}

在吸收塔的设计中,接管的尺寸对气体和液体的传输效率有着直接影响。接管的尺寸取决于流体的流量、流速和系统压力损失等因素。

\subsection{气体入口管道尺寸计算}
根据表中数据,气体流量为 $V = 39.0998 \, \text{kmol/h}$,实际操作气速为 $u_{actual} = 0.436639 \, \text{m/s}$,气体密度为 $\rho_V = 1.5291 \, \text{kg/m}^3$。计算气体体积流量:

\begin{equation}
	V_{\text{体积}} = \frac{V \times M_V}{\rho_V} = \frac{39.0998 \times 31.0566}{1.5291} \approx 795.5 \, \text{m}^3/\text{h}
\end{equation}

转换为 $\text{m}^3/\text{s}$:

\begin{equation}
	V_{\text{体积, s}} = \frac{795.5}{3600} \approx 0.221 \, \text{m}^3/\text{s}
\end{equation}

根据流量公式,气体入口管道直径 $D_{\text{气体进口}}$ 计算如下:

\begin{equation}
	D_{\text{气体进口}} = \sqrt{\frac{4 V_{\text{体积, s}}}{\pi u_{actual}}} = \sqrt{\frac{4 \times 0.221}{\pi \times 0.436639}} \approx 0.448 \, \text{m}
\end{equation}

\subsection{液体入口管道尺寸计算}
根据表中数据,液体流量为 $L = 1587 \, \text{kmol/h}$,液体密度为 $\rho_L = 998.2 \, \text{kg/m}^3$。计算液体体积流量:

\begin{equation}
	L_{\text{体积}} = \frac{L \times M_V}{\rho_L} = \frac{1587 \times 31.0566}{998.2} \approx 49.37 \, \text{m}^3/\text{h}
\end{equation}

转换为 $\text{m}^3/\text{s}$:

\begin{equation}
	L_{\text{体积, s}} = \frac{49.37}{3600} \approx 0.0137 \, \text{m}^3/\text{s}
\end{equation}

假设操作液体流速为 $u_L = 1 \, \text{m/s}$(从表中未给出,使用通常的工业应用速度),则液体入口管道直径 $D_{\text{液体进口}}$ 计算如下:

\begin{equation}
	D_{\text{液体进口}} = \sqrt{\frac{4L_{\text{体积, s}}}{\pi u_L}} = \sqrt{\frac{4 \times 0.0137}{\pi \times 1}} \approx 0.132 \, \text{m}
\end{equation}

\section{离心泵的计算与选择}

离心泵的选型主要基于吸收塔液体的流量和所需扬程。根据表中数据,液体流量为 $L = 28597.7 \, \text{kg/h}$,液体密度为 $\rho_L = 998.2 \, \text{kg/m}^3$。计算液体体积流量:

\begin{equation}
	L_{\text{体积}} = \frac{L}{\rho_L} = \frac{28597.7}{998.2} \approx 28.64 \, \text{m}^3/\text{h}
\end{equation}

扬程计算假设所需扬程为 $H = 30 \, \text{m}$,泵的效率假设为 $\eta_p = 0.75$。离心泵的功率 $P$ 可通过以下公式计算:

\begin{equation}
	P = \frac{\rho_L g L_{\text{体积}} H}{\eta_p} = \frac{998.2 \times 9.81 \times 28.64 \times 30}{0.75} \approx 107374 \, \text{W} \approx 107.374 \, \text{kW}
\end{equation}

\section{风机的选取}

根据表中数据,气体体积流量为 $0.221 \, \text{m}^3/\text{s}$。风机的选取需要确保能够处理该流量。假设风机的效率为 $\eta_f = 0.90$,系统的总压降为 $\Delta P = 2000 \, \text{Pa}$(从表中未给出,使用估计值)。风机的功率 $P_f$ 计算如下:

\begin{equation}
	P_f = \frac{V_{\text{体积, s}} \Delta P}{\eta_f} = \frac{0.221 \times 2000}{0.90} \approx 490.222 \, \text{W} \approx 0.490 \, \text{kW}
\end{equation}